\documentclass[draft]{agujournal2019}

\newcommand{\todo}[1]{\textcolor{red}{\textbf{(#1)}}}

\usepackage{amsmath}

% LTeX: language=en-US

\linenumbers
\journalname{Journal of Advances in Modeling Earth Systems (JAMES)}

\begin{document}

\title{Responses to Humidity and Temperature Perturbations in High-Resolution Simulations of Convection}

\authors{Timothy H. Raupach\affil{1,2}, Chimene L. Daleu\affil{3}, Robert S.
Plant\affil{3}, Steven S.Sherwood\affil{1,2}, and Yi-Ling Hwong\affil{4}}

\affiliation{1}{Climate Change Research Centre, University of New South Wales, Sydney, Australia}
\affiliation{2}{ARC Centre of Excellence for Climate Extremes}
\affiliation{3}{Department of Meteorology, University of Reading, Reading, United Kingdom}
\affiliation{4}{Institute of Science and Technology Austria, Vienna, Austria}

\correspondingauthor{Timothy H. Raupach}{timothy.h.raupach@gmail.com}

\begin{keypoints}
\item \todo{1-3 key points, $\leq$ 140 characters each}
\end{keypoints}

\justifying
\begin{abstract}
\todo{Abstract\ldots}
\end{abstract}

\section*{Plain Language Summary}
\todo{Plain language summary\ldots}

\section{Introduction}

Global and regional climate models (respectively GCMs and RCMs) must take convection into account because of its important atmospheric effects \cite{Manabe_JAS_1964, Wallace_2006}. However, models often use horizontal grid spacings that are much larger than the
horizontal scale of convection, meaning that they are unable to resolve
individual convective cells. In this case, the model must account for the
effects of convection -- such as its heating effect or rain it may produce -- by
parameterizing the sub-grid convection that it can not explicitly resolve. Such
parameterization methods are called convection schemes. 

Many convection schemes have been produced, each with different strengths,
weaknesses, and assumptions \cite<e.g.>{Arakawa_JC_2004,
Rio_CCCR_2019, Lin_AO_2022}. Such are the differences between these schemes, and
so high is the sensitivity of models to convective cloud physics
\cite<e.g.>{Emanuel_JAS_1999}, that convection parameterization is a leading
source of uncertainty in model outputs \cite<e.g.>{Hwong_JAMES_2021}. It is
difficult to compare convection schemes, since traditional methods in which
simulated outputs are compared with observations \cite<e.g.>{Grell_ACP_2014,
Kwon_MWR_2017, Zhang_JC_2017, Zhang_MWR_2011} rely on observation selection and
accuracy \cite{Hwong_JAMES_2021}.

The mathematical framework proposed by \citeA{Kuang_JAS_2010} uses a linear
tangent model to approximate the feedback of small-scale processes on the
large-scale environment, effectively playing the role of a (linear) convective
scheme trained on a large cloud-resolving model (CRM) dataset, such that

\begin{equation} 
\frac{\partial {\bf X}}{\partial t} = {\bf MX}
\end{equation}

where the partial time derivative represents the feedback on the column state
variable {\bf X}, and {\bf M} is a matrix obtained from perturbed runs of a
cloud-resolving model. In this method, the responses of the model (and thus the
convection scheme) are examined in idealized simulations under different small
perturbations to the humidity and temperature tendencies. The idea is that
although there are many non-linear processes in convection, the statistics of
the cumulus ensemble as a whole can be partly described by smooth linear
functions that describe how the ensemble reacts to small changes in its
large-scale environment. \citeA{Kuang_JAS_2010} provided evidence that these
linear responses were sufficient to capture the dynamics of convectively coupled
waves, which evolve slowly compared to typical convective response times and
have relatively small amplitudes. Even if a linear approximation is inadequate
for fully describing convection for more general purposes, it can be a useful
first-order characterization.

To this end, the linear response framework was applied to single column models
(SCMs) by \citeA{Herman_JAMES_2013} specifically for the purpose of testing two
convection schemes against the CRM reference. \citeA{Hwong_JAMES_2021} used the framework to evaluate
convection, planetary boundary, and microphysics schemes' responses to
perturbations in SCMs. They showed that the results isolate the impact of the
convection scheme in at least the tested SCMs, and examined many schemes in
actual use. The technique allows for efficient comparison of different model
setups, and identified a wide range of responses among current schemes, thus
showing the potential of this test, but does not answer the question of which
setup is most accurate. To answer this question we need a "truth" response,
which is not available from observations. A study of the linear responses of a
model that uses the least possible parameterization is required, along with robustness tests. \citeA{Kuang_JAS_2010}
evaluated his linear responses using rather coarse resolution (2 km)  runs of
the System for Atmospheric Modelling \cite<SAM, e.g.>{Khairoutdinov_JAS_2003},
and in a relatively small domain (128 km square). This resolution would not be
considered large-eddy resolving by today's standards, and the experiment has not
been repeated in any other model.

In this study, we repeated the above perturbation experiments on two different
models across a range of CRM and large eddy simulation (LES) resolutions,
seeking to produce benchmark test results. This also enabled us to test the
conclusion of \citeA{Hwong_JAMES_2021} that the test results are not sensitive
to treatment of the planetary boundary layer (PBL), when convection is resolved
explicitly and the PBL dynamics might interact more directly with it. The key 
question is whether the linear responses are robust, or sensitive to decisions in 
high-resolution modelling such as the grid length, model dynamical core, or 
treatment of microphysics or eddy mixing; in particular, can we establish a 
reference ``truth'' that is less uncertain than the variations between convection 
schemes themselves? A related question is whether the response is dominated by the 
modelled convection itself, or whether it might be sensitive to the specification of 
surface properties in an idealized modelling setting. A third question addressed 
here, which is becoming more relevant as we enter an age of global CRMs, is what 
grid resolution they would need to properly represent convective-scale feedbacks on 
large scale flows, arguably the key benefit of global CRMs, and how sensitive the 
feedback might be to parameterizations remaining in those models or to the numerics 
or other details of the CRM. In summary, can we judge differences between model 
schemes as reported in \citeA{Hwong_JAMES_2021} using LES or CRM models?

\section{Methods}
\label{sec:methods}

We used two models in this work: the Advanced Research (AR) version of the
Weather Research and Forecasting (WRF) model, version 4.1.4
\cite{Skamarock_2019}, and the Met Office/NERC Cloud Model
\cite<MONC,>{Brown_2020} as used in \citeA{Daleu_QJRMS_2023}.

\subsection{Common Model Settings}

Both models were used to simulate convection at high resolution over a water
surface with periodic boundary conditions. The models used a constant sea
surface temperature (SST) of 301.15 K. This value is a little different from
that used in \citeA{Kuang_JAS_2010} but matches \citeA{Hwong_JAMES_2021}. We
used idealized radiation as in \citeA{Kuang_JAS_2010}, \citeA{Herman_JAMES_2013} and \citeA{Hwong_JAMES_2021}. Following \citeA{Herman_JAMES_2013}, the radiative cooling tendency
was $\partial \theta_{rad} / \partial t = \theta_t/\Pi$ K day$^{-1}$, where 
$\Pi$ is the Exner function that converts temperature to potential temperature, and 
$\theta_t$ is a temperature tendency set as \todo{double check in code for WRF}

\begin{align}
 \theta_t = \begin{cases}
    -1.5\; \textrm{K day}^{-1} & \textrm{if}\, p \geq 200\; \textrm{hPa}, \\
    0\; \textrm{K day}^{-1} & \textrm{if}\, p \leq 100\; \textrm{hPa}, \\
    -1.5 + \frac{1.5 (p-100)}{100}\; \textrm{K day}^{-1} & \textrm{if}\, 100 < p < 200\; \textrm{hPa}. \\
 %* $t$'s value varies linearly between -1.5 and 0 K day$^{-1}$ from 100 to 200 hPa.
 \end{cases}
\end{align}

Our treatment of surface fluxes and winds also follows that of \citeA{Hwong_JAMES_2021}, which differed slightly from that of \citeA{Kuang_JAS_2010}. First, we used idealized evaporation as in \citeA{Chua_GRL_2019}, with minor
differences. We used a constant surface wind speed of $W_s = 4.8$ m s$^{-1}$,
following \citeA{Hwong_JAMES_2021} who used the same approach with a fixed value
of 0.001 for the surface exchange coefficient. While the wind speed here and in
\citeA{Hwong_JAMES_2021} was slightly different from that of
\citeA{Kuang_JAS_2010}, the important quantity is the product of the wind with
the surface exchange coefficient, and the exchange coefficient value in
\citeA{Kuang_JAS_2010} was not stated. Here we used a fixed value of 0.005 for
this quantity \todo{double check for WRF}. Level-mean horizontal winds
were relaxed to $u = 0$ m s$^{-1}$ and $v = 5$ m s$^{-1}$, where $u$ is the
zonal and $v$ is the meridional wind component, respectively, with a three-hour
relaxation time.

Following \cite{Kuang_JAS_2010} \todo{check}, for each resolution, the models were run until they reached radiative convective
equilibrium (RCE) at time $t_1$. After RCE was reached, small perturbations in
the imposed temperature or moisture forcings were introduced, while a control
was run with no perturbations, and the models were allowed to again reach RCE at
time $t_2$. For a time block after $t_2$, domain-mean profiles in the control
runs were subtracted from the  domain-mean atmospheric profiles for
the perturbed runs \todo{check order in code} to examine the differences made by the perturbations. In the following two
sections we examine the model-specific setups for WRF and MONC, respectively.

\subsection{WRF}

WRF was run on a 20 $\times$ 20 km$^2$ domain for horizontal grid spacings of 4
km, 1 km and 100 m. The runs at 4 km and 1 km grid resolution used 74 grid
points in the vertical while the run at 100 m grid spacing used 370 vertical
grid points. The parameterizations used are summarized in Table
\ref{tab:WRF_schemes}. The model time step was set to 6 s for 4 km and 1 km grid
spacing and 1 s for 100 m grid spacing, and outputs were recorded hourly for the
1 km and 4 km cases, and at 2 hour resolution for the 100 m cases. For the runs
with 1 km and 4 km grid spacing, the models were initialized using a
Radiative-Convective Equilibrium Model Intercomparison Project (RCEMIP) profile
\cite{Wing_GMD_2018}. For the runs at 100 m grid spacing, the model was
initialized using the mean RCE model state from the 1 km control run, so that
the LES runs started near RCE to reduce computation time. Temperature and water
vapor mixing ratio in the stratosphere were relaxed to the initial profile
values to avoid model drift, as suggested by \citeA{Herman_JAMES_2013}.
Relaxation was applied above 160 hPa with the inverse relaxation constant
increasing linearly from $1/\tau = 0$ day$^{-1}$ at 160 hPa to $1/\tau = 0.5$
day$^{-1}$ at and above 100 hPa \cite{Herman_JAMES_2013}. Full diffusion
was used at all resolutions; at 4 km and 1 km
resolutions, horizontal diffusion was diagnosed from deformation and vertical
diffusion was assumed done by the planetary boundary layer scheme
(\texttt{km\_opt = 4}) while the simulations at 100 m used a prognostic equation
(\texttt{km\_opt = 2}) for turbulent kinetic energy \cite{Skamarock_2019}.

\begin{table}[t]
    \caption{Parameterizations used in the WRF simulations. These schemes were
     used for all resolutions, except the boundary layer scheme which was
     enabled only for the 4 km and 1 km simulations. No convection
     parameterization was used.}
    \label{tab:WRF_schemes}
    \centering
    \begin{tabular}{lll}
    \hline
    \textbf{Parameterization} & \textbf{Scheme} & \textbf{Reference} \\
    \hline
    Microphysics & Thompson & \citeA{Thompson_MWR_2008} \\
    Boundary layer & YSU & \citeA{Hong_MWR_2006} \\
    Surface layer & Revised MM5 & \citeA{Jimenez_MWR_2012} \\
    \hline
    % \multicolumn{2}{l}{$^{a}$Footnote text here.}
    \end{tabular}
\end{table}

We used idealized evaporation as in \citeA{Chua_GRL_2019} and
\citeA{Hwong_JAMES_2021}, with a fixed surface wind speed and the drag
coefficient assumed to be 0.001. Then surface heat flux (SH) was calculated
using modified Equation 1 from \citeA{Chua_GRL_2019}, such that

$$
\textrm{SH} = 0.001 \rho_a W_s c_p (T_s - T_a),
$$

\noindent where $\rho_a$ (kg m$^{-3}$) is the near surface air density, $T_s$
(K) is surface temperature (we use SST), and $T_a$ (K) is the near-surface air
temperature. \todo{Check $W_s$ is same as in Eq 2.} We used 

$$
T_a = T_l \left(\frac{p_s}{p_l}\right)^{\frac{R_d}{c_p}},
$$

\noindent where $T_l$ (K) is the temperature at the first model level above the
surface, $p_s$ (hPa) is surface pressure, $p_l$ (hPa) is pressure at the first
model level, $c_p$ (J kg$^{-1}$ K$^{-1}$) is the specific heat capacity of dry
air at constant pressure, and $R_d$ (J kg$^{-1}$ K$^{-1}$) is the gas constant
for dry air. Latent heat flux (LH) was calculated using modified Equation 2 from
\citeA{Chua_GRL_2019}, such that

$$
\textrm{LH} = 0.001 \rho_a W L (q_{sat} - q_a),
$$

\noindent where $L$ is the typical latent heat of vaporization of water,
$q_{sat}$ is the saturated water vapor mixing ratio at $T_s$ (we use WRF's
``ground saturated mixing ratio'') and $q_a$ is the water vapor mixing ratio at
the lowest model level. Note that while \citeA{Chua_GRL_2019} assumed that air
density near the surface is 1 kg m$^{-3}$, we used the near-surface density
value produced by WRF (around 1.16 kg m$^{-3}$).

Because the pressure levels in the WRF model were not exactly at the points
around which we want to perturb, we adapted the form of perturbations used by
\citeA{Herman_JAMES_2013} so that only the Gaussian portion of the perturbation
function is used, and instead of selecting a level $k$ a pressure $p_p$ (hPa)
around which to perturb is selected. The perturbation function used in the WRF
simulations for the $i$th model level was

$$
f_i = \exp\left[ - \left( \frac{p_p - p_i}{75 \textrm{hPa}}\right)^2 \right],
$$

\noindent where $p_p$ (hPa) is the chosen perturbation pressure and $p_i$ is the
pressure at level $i$. We calculated the spatial means per interpolated model
pressure level, per time step. The temporal average of these mean profiles over
the RCE period was calculated for each run for comparison.

\subsection{MONC}

MONC was run on a 64 $\times$ 64 km$^2$ domain for grid spacings of 1 km, 500 m,
and 250 m. The MONC setup was similar to that used in \citeA{Daleu_QJRMS_2023}:
the domain height was 20 km, with 85 levels on a stretched grid so there were more levels near the surface, and a
Newtonian damping layer above 16 km. We used bi-periodic lateral boundary
conditions. The 3D Smagorinsky scheme was used to calculate subgrid turbulent
eddy fluxes, and the Cloud AeroSol Interactive Microphysics \cite<CASIM,
see>{field23} scheme
represented cloud processes. The model time step is variable in MONC and ranged between 1 and 2 seconds.

Perturbations were made using a combination of the delta and Gaussian functions,
as in Equation~4 of \citeA{Herman_JAMES_2013}. We used an idealized treatment of
evaporation in the sense that a constant near-surface wind speed of 5
m\,s$^{-1}$ was imposed \todo{confirm if 5 or 4.8 for WRF}. For pressure levels
less than 160 hPa, temperature and moisture fields were relaxed to the values of
a previous RCE run, following \cite{Herman_JAMES_2013} \todo{check if same for
WRF and if so move to common settings}.

\subsection{Perturbations to Temperature and Humidity}

Constant perturbations were applied to potential temperature or water vapor
tendencies \todo{RSP: Text says tendencies but the numbers in the Table are in units of K and kg/kg rather than (say) K/d?} at given model levels. Given computational constrains, not all
perturbations were applied in all runs.  Table \ref{tab:pert_runs} shows which
perturbations were applied to which model runs. For the WRF runs, we used the
results of \citeA{Hwong_JAMES_2021} to find two levels on which perturbations 
provided the most information about perturbations across other levels (not shown 
here). The two ``most representative'' levels were a low (850 hPa) and a high (412 
hPa) perturbation level, which we use here in the high-resolution WRF runs. Other
levels on which perturbations were applied were 415 hPa (MONC only), 500 hPa,
600 hPa, and 730 hPa. For simplicity of comparison, in the following we refer to
the MONC results for a perturbation at 415 hPa as results for a perturbation at
412 hPa.

{
\begin{table}
    \centering
    \caption{Perturbations applied by variable, model, and grid spacing. Here,
    412 hPa refers to 412 hPa in the WRF runs but 415 hPa in the MONC runs. A
    $\bullet{}$ symbol indicates the given model was run with the given
    resolution and perturbation. $T$ is \todo{potential} temperature in K and
    $q$ is water vapor mixing ratio in kg kg$^{-1}$.}
    \label{tab:pert_runs}
    \renewcommand{\arraystretch}{0.6}
    \begin{tabular}{lllccccc}
        \multicolumn{3}{r}{\textbf{Model}} & \multicolumn{2}{c}{\textbf{WRF}} & \multicolumn{3}{c}{\textbf{MONC}} \\
        \multicolumn{3}{r}{\textbf{Grid spacing}} & 4 and 1 km & 100 m & 1 km & 500 m & 250 m \\
        
        \textbf{Variable} & \textbf{Perturbation} & \textbf{Level} & & & & & \\
        \hline
        % T +
        T & $+0.5$ K & 412 hPa & $\bullet{}$ & $\bullet{}$ & $\bullet{}$ &  & $\bullet{}$ \\
        & & 500 hPa & $\bullet{}$ & & $\bullet{}$ & $\bullet{}$ & $\bullet{}$ \\
        & & 600 hPa & $\bullet{}$ & & $\bullet{}$ &  & $\bullet{}$ \\
        & & 730 hPa & $\bullet{}$ & & $\bullet{}$ &  & $\bullet{}$ \\
        & & 850 hPa & $\bullet{}$ & $\bullet{}$ & $\bullet{}$ & $\bullet{}$ & $\bullet{}$ \\
        % T -
        & $-0.5$ K & 412 hPa & $\bullet{}$ & $\bullet{}$ & $\bullet{}$ &  & $\bullet{}$ \\
        & & 500 hPa & $\bullet{}$ & & $\bullet{}$ &  & $\bullet{}$\\
        & & 600 hPa & $\bullet{}$ & & $\bullet{}$ &  & $\bullet{}$ \\
        & & 730 hPa & $\bullet{}$ & & $\bullet{}$ &  & $\bullet{}$ \\
        & & 850 hPa & $\bullet{}$ & $\bullet{}$ & $\bullet{}$ &  & $\bullet{}$ \\
        \hline
        % q +
        q & $+0.0002$ kg kg$^{-1}$ & 412 hPa & $\bullet{}$ & $\bullet{}$ & $\bullet{}$ &  &  \\
        & & 500 hPa & $\bullet{}$ & & $\bullet{}$ & $\bullet{}$ & $\bullet{}$ \\
        & & 600 hPa & $\bullet{}$ & & $\bullet{}$ &  &  \\
        & & 730 hPa & $\bullet{}$ & & $\bullet{}$ &  &  \\
        & & 850 hPa & $\bullet{}$ & $\bullet{}$ & $\bullet{}$ & $\bullet{}$ & $\bullet{}$ \\
        % q -
        & $-0.0002$ kg kg$^{-1}$ & 412 hPa & $\bullet{}$ & $\bullet{}$ & $\bullet{}$ &  &  \\
        & & 500 hPa & $\bullet{}$ & & $\bullet{}$ &  &  \\
        & & 600 hPa & $\bullet{}$ & & $\bullet{}$ &  &  \\
        & & 730 hPa & $\bullet{}$ & & $\bullet{}$ &  &  \\
        & & 850 hPa & $\bullet{}$ & $\bullet{}$ & $\bullet{}$ &  &  \\
        \hline
    \end{tabular}
\end{table}
}

\section{Results}
\label{sec:results}

\subsection{Radiative-Convective Equilibrium Mean State}

We used plots of spatially-averaged precipitable water by time to determine when
the models had reached RCE. Figure \ref{fig:rce_pw} shows precipitable water by
time for the WRF runs and columnar water vapor by time for the MONC runs \todo{RSP: isn't this the same thing?}. For
the WRF runs, at 4 km grid spacing the RCE state had a precipitable water (PW)
of $\sim$40 mm, which rose to $\sim$44 mm at 1 km grid spacing and $\sim$42 mm
at 100 m grid spacing. For the MONC runs, \todo{\ldots}.

\begin{figure}[pth]
    \noindent\includegraphics[width=\textwidth]{figures/runs_timeseries.pdf}
    \caption{Time series of water content in WRF and MONC. The plots show time
    series of spatial mean column water vapor (CWV) by grid spacing. RCE runs
    were computed until precipitable water content/columnar water content
    stabilized, and the model was at RCE. At this point the perturbations were
    introduced, and the model was run until all perturbed runs were also at RCE.
    The parts of the time axes highlighted in green show the ``RCE region'' over
    which mean profiles were calculated for comparison. For the MONC runs, the
    highlighted region shows the maximum range of RCE regions, since RCE regions
    were defined as the last 20 days in each simulation for 1 km runs and the
    last 10 days in each simulation for the 500 m and 250 m runs. Note that to
    reduce the number of colors required, positive and negative perturbations
    per variable and perturbation pressure use the same color.}
    \label{fig:rce_pw}
\end{figure}

Mean profiles from the control run in RCE are shown in Figure
\ref{fig:rce_profiles}. The mean states are similar across the two models \todo{YLH: This is already quite interesting - for the SCMs in Hwong et al (2021) the mean states have a huge spread (even though we also used idealized radiation and evaporation): this means simply by resolving convection explicitly we already bring the mean states closer to each other. Perhaps worth mentioning this in one sentence.},
especially within the context of larger inter-model comparisons of RCE states
\todo{cite examples}. To some extent the similarities may be due to the
idealized treatment of radiation and evaporation used here. Moisture content
generally increases with increasing resolution, although the changes are modest
at 1 km and higher-resolution grid spacing. MONC is marginally more moist than
WRF. The additional moisture at finer resolutions appears in the lower part of
the free troposphere above about 900 hPa, which we hypothesize is the result of
some transport from shallow convection. Mean wind profiles for the 100 m run,
which had increased vertical resolution, and was temporally shorter than
the other runs, are not as smooth as those for the 1 km and 4 km runs. At 100 m
grid spacing the profile of relative humidity is fairly uniform with height
between $\sim$950 and $\sim$600 hPa, whereas at 1 km and 4 km grid spacing this
part of the atmosphere was drier in the WRF runs, with the 4 km WRF simulation
drying to $\sim$60\% relative humidity at about 800 hPa. The MONC runs also show
decreased relative humidity with decreased resolution \todo{SS: avoid this terminology (what does it mean?) -- say decreased/increased grid sizes, or coarser/finer resolution}, and are drier than the
WRF runs in a region around $\sim$600 hPa.

\begin{figure}[pth]
    \noindent\includegraphics[width=\textwidth]{figures/rce_profiles}
    \caption{Mean profiles for selected variables in the control runs' RCE
    periods, by grid spacing. The water vapor mixing ratio x-axis is truncated
    at 18 g kg$^{-1}$. All plots stop at 200 hPa, meaning that a cirrus layer
    that forms at the tropopause in the MONC simulations is not shown here.
    Temperatures are shown as anomalies from a reference moist pseudo-adiabat
    calculated using a lowest temperature of 300 K.}
    \label{fig:rce_profiles}
\end{figure}

Hydrometeor profiles, also shown in Figure \ref{fig:rce_profiles}, are sensitive
to resolution. For the WRF runs, finer resolution resulted in more rain water
below $\sim$650 hPa and less above, more cloud droplets and snow, more graupel below $\sim$500 hPa and less above,
and mixed changes in ice content \todo{RSP: The language used here is fine if the WRF microphysics is single moment, but will be confusing if it is double moment. It would be a good idea to confirm that explicitly in the model descriptions.}. The MONC runs showed marginally less
sensitivity, but the character of the changes was broadly the same as in WRF, except that MONC at finer resolution produced
more rain at
all levels. Profile shapes were \todo{SS: use consistent tense (past vs present)} broadly similar between WRF and MONC, with
perhaps the strongest difference in liquid water, for which MONC produced
significantly more at $\sim$800 hPa and less above $\sim$600 hPa. There was a large change in the liquid water content between the 4 km and 1 km grids
in WRF, and a particularly marked difference in WRF between the 1 km and 100 m grids around $\sim$600 hPa. Graupel and
snow are found at lower heights in WRF than in MONC. In both models finer resolution brings more snow at cloud top and less graupel, with relative differences varying by height. It is unclear
whether these microphysical changes are simply a consequence of mean
moisture changes or whether they reflect the impact of cloud
turbulence differences.

\subsection{Convective Organization}

\todo{RSP: Perhaps needs a small intro to say why we did this. i.e. to check whether any of the model-model differences or resolution chnages might be due to some setups developing organiztion.} We tested for convective organization by examining the spatial variance of
precipitable water scaled by its spatial mean (in WRF, not shown), and by
examining snapshots of precipitation cross-sections inside the equilibrium
period (in MONC, not shown) \todo{RSP: Chimene tested the same metric as you used for WRF. She also looked at the "subsidence fraction" which is the fractional area of the domain where the vertically integrated mass-weighted vertical wind is -ve. The latter was about 55\% but it would be considerably larger in an aggregated state} and found no evidence of meaningful organization in
these fixed-radiation setups. Our domain sizes are chosen such that they are too
small for self-aggregation of convection to be expected \cite{Muller_JAS_2012}.
\todo{SS: Did Muller and Held use idealized radiation?  If not then their
conclusions would not apply here and organization might not be expected at any
domain size.} \todo{YLH: They use interactive radiation so yes, their conclusions are not really relevant here. I think it is sufficient to say there is no organization because we used fixed radiation everywhere. If you would like to cite something, you can cite Muller and Bony (2015) that shows imposing radation can lead to organization provided two contrasting profiles are imposed in dry and moist regions (which we did not do, so organization is not expected).}

\subsection{Responses to Heating (Temperature) Perturbations}

The responses of the models to perturbations in temperature at 412/415 hPa and
850 hPa are shown in Figures \ref{fig:tpert_412} and \ref{fig:tpert_850},
respectively, with results for perturbations at other levels (500, 600, and 730
hPa) shown in the Supplement (Figures \ref{fig:tpert_500}, \ref{fig:tpert_600},
and \ref{fig:tpert_730}). All responses to these perturbations are much smaller
than the differences in the mean profiles owing to resolution (Figure
\ref{fig:rce_profiles}). WRF and MONC produce responses of similar amplitude. In
the temperature, water vapor mixing ratio, and relative humidity fields, MONC is
more likely to show nonlinearity, or differences in responses to positive and
negative perturbations, than WRF at 1 km or 4 km resolutions, with the strongest
example being the temperature response in the upper troposphere to a temperature
perturbation at 850 hPa. The 100 m WRF results also show some
nonlinearity in their responses. In the hydrometeor fields, the strongest
nonlinearity is found in the WRF results at 100 m grid spacing. However,
there was a large range in responses at 100 m grid spacing (Figures
\ref{fig:var_T_412} and \ref{fig:var_T_850}), meaning that the discrepancies
between responses for positive and negative perturbations are less likely to be
caused by nonlinearity than by noise in the response signal for these
hydrometeor mixing ratios.

\begin{figure}[pth]
    \noindent\includegraphics[width=\textwidth]{figures/pert_diffs_T_0.5_@412}
    \caption{Differences in model values by pressure, after a perturbation of
    +0.5 K was introduced in the potential temperature tendency field at 412 hPa
    (415 hPa for MONC). The solid vertical black line shows zero difference. The
    dashed horizontal line shows the approximate level of maximum perturbation.
    Responses to positive perturbations are shown with opaque lines, while
    responses to negative perturbations are shown with semi-transparent lines.
    Responses to negative perturbations have been multiplied by $-1$, so they
    overlay responses to positive perturbations if the positive and negative
    responses are symmetrical. Black circles show the responses to the same
    perturbation recorded by \citeA{Kuang_JAS_2010}.}
    \label{fig:tpert_412}
\end{figure}

\begin{figure}[pth]
    \noindent\includegraphics[width=\textwidth]{figures/pert_diffs_T_0.5_@850}
    \caption{As in Figure \ref{fig:tpert_412}, but for a temperature
    perturbation at 850 hPa.}
    \label{fig:tpert_850}
\end{figure}

\subsubsection{Temperature Responses}

In response to temperature perturbations, the temperature response tends to
increase with height from the surface to a maximum just below the perturbation
level, then decreases around the perturbation level before once more increasing
with height. This structure is most apparent in the responses to perturbations
in the middle levels (500, 600, 730 hPa), and it is not apparent in the
responses for a perturbation at the low level of 850 hPa. This ``dog leg''
structure is also often more marked in MONC than in WRF which gives a smoother
temperature response in the vertical. The sub-km resolution runs with MONC also
produce smoother responses than the 1 km runs.

\subsubsection{Moisture Responses}

The moisture responses to the temperature perturbations again appear smoother vertically from WRF than from MONC \todo{SS: Is that all?  Can we add to this sentence to make it a statement of the key outcomes?}. The temperature perturbations at
$\sim$412 and 850 hPa produce the strongest differences among the various runs.
In some cases (WRF at 4 km grid spacing and MONC at 1 km and 500 m), the 850-hPa perturbation
can induce a drying response at and just above the perturbation level. With WRF,
this response becomes a moistening at finer resolution. With MONC, the drying is
a little less marked at 500 m than at 1 km grids, and at 250 m the
response was zero at this level. For the temperature perturbation at 850 hPa,
moistening in the boundary layer increases for coarser resolutions in WRF, and
to a lesser extent for MONC. A temperature perturbation at 412 hPa produces a
minimum in the moistening in the WRF runs at around 900 hPa. In MONC there is a
similarly pronounced minimum a little higher at about 850 hPa seen in the 1 km
run. Similar comments apply for 500 and 600 hPa as for 412 hPa. There is often a
small minimum in the moistening responses around or at the perturbation level
itself, particularly for perturbations applied at lower levels (at higher levels
a second minimum tends to appear below the perturbation). For a
temperature perturbation at 730 hPa, the moistening minimum around the
perturbation level is pronounced in MONC and reaches around zero, but it does not
produce the drying seen when the perturbation is applied at 850 hPa. One way to
think about the drying with perturbations at 850 hPa is that the minimum we see
consistently just above the boundary layer combines with the minimum that we
often see at or below the perturbation level.

\subsubsection{Hydrometeor Responses}

\todo{SS: Since this paragraph is pretty heavy, instead of starting by announcing what you are going to look at, can you lead off with the takeaway message?  The rest of the paragraph then fills in the details but knowing where it is going makes them much easier to digest.} We now turn to how hydrometeor content responds to heating perturbations (keeping in mind the high variability in the 100-m WRF
hydrometeor responses). For the temperature perturbation at 850 hPa in WRF, there
is generally a reduction of rain below the freezing level; MONC shows a stronger
reduction than in WRF below the freezing level but an increase around the
freezing level, which is a feature not seen in the WRF results. Except for a
near-surface increase, a reduction in cloud liquid water also occurs, and this
is over a deeper layer in WRF than in MONC. There is also some reduction to snow
and increase in graupel at upper levels. For WRF runs with temperature
perturbations at lower levels, the profiles of liquid water changes have
reductions with minima that are aligned with or just above the perturbation
height. As the perturbation level increases in the WRF runs, a maximum, in liquid
water changes appears between the perturbation level and the surface. The 4 km
WRF runs in particular show a sharp increase in cloud water at about 900 hPa
under all temperature perturbations. The responses of hydrometeors in MONC have
a broadly similar character, but some of the amplitudes are quite different.
There are similar amplitudes to WRF for rain, but weaker responses in liquid
water, ice and snow. MONC responded more strongly for graupel, however,
especially in response to perturbations at 412 and 500 hPa. A positive
temperature perturbation generally reduces graupel content except for some
increases at high levels when the perturbation was closer to the surface. There
are some changes in responses with grid length. For the WRF runs the rain
changes tend to increase in magnitude with increasing resolution, while the MONC
responses are more similar. For MONC the liquid water reductions are stronger at
250 m than at 1 km spacings, and for temperature perturbations at lower levels
snow is also more strongly reduced at 250 m than at 1 km grid spacing. The
changes in graupel become weaker with increasing resolution in MONC runs yet
stronger with increasing resolution for WRF runs. 

\subsection{Responses to Moistening (Humidity) Perturbations}

The responses of the models to perturbations in moisture at 412/415 and 850 hPa
are shown in Figures \ref{fig:qpert_412} and \ref{fig:qpert_850}, respectively,
while perturbations for 500 hPa, 600 hPa, and 730 hPa, are shown in Figures
\ref{fig:qpert_500}, \ref{fig:qpert_600}, and \ref{fig:qpert_730}, respectively.
Several elements of these results are similar to the results for temperature
perturbations: moisture and hydrometeor responses to moisture perturbations are
again smaller than the differences in the mean profiles due to resolution \todo{SS: How can these be compared, they are in different units (one is e.g. K while the other is in K per K/day e.g. units of time).   Since the perturbations are intentionally small, we would not expect responses to them to be very large compared to anything.} \todo{RSP: Isn't the response a difference in the mean state when the strength of forcing is perturbed? So that will be in K along with the difference in mean state due to resolution}; WRF
and MONC produced responses of similar amplitude, except for cloud liquid water
and ice where the changes in MONC were much smaller; and MONC shows more
nonlinearity in temperature, water vapor and relative humidity responses than
WRF does at 1 km and 4 km grid spacings. MONC relative humidity seems to be
particularly prone to non-linear responses. Similar to the responses to the
temperature perturbations, the strongest apparent nonlinearity after moisture
perturbations is in the hydrometeor fields and especially for the 100 m grid
spacing WRF responses, but as for the temperature perturbations there was a
large range in responses (Figures \ref{fig:var_q_412} and \ref{fig:var_q_850}).

\begin{figure}[pth]
    \noindent\includegraphics[width=\textwidth]{figures/pert_diffs_q_0.0002_@412}
    \caption{As for Figure \ref{fig:tpert_412}, but for differences in model
    values by pressure, after a perturbation of 0.2 g kg$^{-1}$ was introduced
    in the water vapor mixing ratio tendency field at 412 hPa.}
    \label{fig:qpert_412}
\end{figure}

\begin{figure}[pth]
    \noindent\includegraphics[width=\textwidth]{figures/pert_diffs_q_0.0002_@850}
    \caption{As in Figure \ref{fig:qpert_412}, but for a water vapor mixing
    ratio perturbation at 850 hPa.}
    \label{fig:qpert_850}
\end{figure}

\subsubsection{Temperature Responses}

With moisture perturbations, temperature increases in the boundary layer are
very consistent between resolutions in WRF, but the free tropospheric
temperature increases are stronger at coarser resolution and less strong for 100-m grid spacing across the whole profile. Particularly notable is the increase in
temperature increase between 1 km and 4 km grid spacings for a perturbation at
412 hPa. MONC responses show a similar response below about 600 hPa and above
about 850 hPa, but outside this region the responses for higher resolution runs
are stronger than those for 1-km runs. The boundary-layer temperature increases
are very consistent between WRF and MONC. The free-tropospheric increases are
stronger in MONC for a low-level moisture perturbation and weaker for an
upper-level moisture perturbation.

\subsubsection{Moisture Responses}

The WRF responses of specific humidity to moisture perturbations have minima at
around 900 hPa and to a lesser extent around 650 hPa \todo{SS: regardless of the forcing level?  State this}. These minima are modest at
fine grid spacing, but at coarse resolution the lower minimum becomes stronger
even for perturbations made at upper levels. It does however remain positive for positive perturbations. Such minima are also present in MONC, the
lower level one being a little higher and much more pronounced than in WRF at
the same grid spacings, and the MONC responses also remaining positive for
positive perturbations.

\subsubsection{Hydrometeor Responses}

There are nonlinearities in the hydrometeor responses, with the most obvious
case that of rain mixing ratios. With some exceptions, increased moisture
tendency shows a small reduction in cloud liquid water across the whole profile
but an increase close to the surface, reductions in ice in the high troposphere,
and reductions in snow above the perturbation level. Graupel is consistently
strongly reduced around 600 hPa, but there are responses with opposite signs in
the upper levels. The responses of hydrometeors in MONC are generally weaker
than WRF for liquid water, ice and snow. MONC responds more strongly for
graupel, however, especially in response to higher level perturbations where
nonlinearities are evident.

\section{Conclusions}
\label{sec:conclusions}

\begin{itemize}
\item \todo{Hydrometeor responses show much more noise than temperature, water
vapor mixing ratio, and relative humidity responses.}
\item \todo{Steve: We need to think about what the main outcomes are (say, our
three bullets). The conclusion that I think would be most interesting to the
broader community is the resolution needed for each model to approach apparent
convergence of these time-averaged responses (noting that it isn’t a rigorous
convergence test due to computational limitations).  The result suggests that
for most models, to simulate convectively coupled dynamics correctly will likely
require better resolution than is currently used in e.g. DYAMOND \todo{ref}. YLH: Yes, agree that a powerful and interesting conclusion would be to say at x km resolution the models appear to converge. One way to structure the Conclusion section might be to answer the main RQs laid out in the Introduction section (which are really interesting, i think). RSP: Agree that this is a key conclusion. Worth reminding the reader too that we are looking at perturbed forcing on an RCE base, so conclusions apply for deep convection. Mid-level or shallow convection woud no doubt need higher resolutions.}
\item \todo{Steve: Another conclusion is that tangent linear responses become relatively similar
across several LES models once this resolution is reached, and so can be used as
ground truth for testing GCM schemes. RSP: This seems a fair conclusion for the 2 models we have running at different resolution. But it will be good to set this in the context of...}
\begin{enumerate}
\item \todo{the difference between the two models (ie, each running at their highest resolution). If we can agree that the 2 models at their highest resolutions tell us the truth with an error bar, then a good question to ask is whether we can run one or other of the two models at a lower resolution ok and still be within truth + error bar}
\item \todo{We haven't said much about the Kuang results in the results discussion, but it would be good to comment on that at the end. How do his single-resolution results compare to our highest resolution results? Do we think that they provide a good enough "truth" or do we now think that are likely deficient in some ways?}
\item \todo{We can reflect back on the SCM / parameterized study. An important points will be to say that the differences between responses for different parameterizations are larger than any uncertainties that we believe are present in defining the "truth" response. Also, the conclusions here are not the right place to re-evaluate everything in that paper, but it might be useful to finish this new paper by commenting on one or two example responses seen there. i.e. in the earlier paper we could only say that scheme X had such and such a response that was unusual or a little suspicious but now that we have done the new study we can now say that we think this type of response from scheme X is actually just plain wrong.}
\end{enumerate}
\item \todo{Steve: Would be good to work through the results
to see how confident we are in such findings, how important each one is, and if
there are other key findings (another one you mention is the relative robustness
of different types of response e.g. T vs q at different levels). RSP: Yes, this seems important. We can try to summarize our overall impressions re convergence and differences between LES and GCMs etc but there clearly will be some responses that are much more robust than others. We don't want to go through everything in the results all over again, but we should be able to offer some general comments on which variables and levels are more or less robust. It might be best to do that at the beginning of the conclusions? Then we can move on to saying that for those responses which are robust, then these are our conclusions about the relative sizes of resolutions differences, LES differences and GCM scheme differences}
\end{itemize}

%% Enter Figures and Tables near as possible to where they are first mentioned:
%
% DO NOT USE \psfrag or \subfigure commands.
%
% Figure captions go below the figure.
% Table titles go above tables;  other caption information
%  should be placed in last line of the table, using
% \multicolumn2l{$^a$ This is a table note.}

% \begin{figure}
% \noindent\includegraphics[width=\textwidth]{anothersample.png}
% \caption{caption}
% \label{pngfiguresample}
% \end{figure}

% Acronyms
%  \begin{acronyms}
%  \acro{Acronym}
%  Definition here
%  \acro{EMOS}
%  Ensemble model output statistics
%  \acro{ECMWF}
%  Centre for Medium-Range Weather Forecasts
%  \end{acronyms}

\section{Open Research}

% AGU requires an Availability Statement for the underlying data needed to
% understand, evaluate, and build upon the reported research at the time of peer
% review and publication.

% Authors should include an Availability Statement for the software that has a
% significant impact on the research. Details and templates are in the
% Availability Statement section of the Data and Software for Authors Guidance:
% \url{https://www.agu.org/Publish-with-AGU/Publish/Author-Resources/Data-and-Software-for-Authors#availability}

% It is important to cite individual datasets in this section and, and they must
% be included in your bibliography. Please use the type field in your bibtex file
% to specify the type of data cited. Options include [Dataset], [Software],
% [ComputationalNotebook], [Collection].
% Example:
%
%@misc{https://doi.org/10.7283/633e-1497,
%  doi = {10.7283/633E-1497},
%  url = {https://www.unavco.org/data/doi/10.7283/633E-1497},
%  author = {de Zeeuw-van Dalfsen, Elske and Sleeman, Reinoud},
%  title = {KNMI Dutch Antilles GPS Network - SAB1-St_Johns_Saba_NA P.S.},
%  publisher = {UNAVCO, Inc.},
%  year = {2019},
%  type = {dataset}
%}

\bibliography{library}

\newpage
\section*{Appendix}
\setcounter{figure}{0}
\renewcommand{\thefigure}{A\arabic{figure}}

\begin{figure}[pth]
    \noindent\includegraphics[width=\textwidth]{figures/pert_diffs_T_0.5_@500}
    \caption{As in Figure \ref{fig:tpert_412}, but for a temperature
    perturbation at 500 hPa, and with black circles showing the responses of
    \citeA{Kuang_JAS_2010} to a temperature perturbation at 483 hPa.}
    \label{fig:tpert_500}
\end{figure}

\begin{figure}[pth]
    \noindent\includegraphics[width=\textwidth]{figures/pert_diffs_T_0.5_@600}
    \caption{As in Figure \ref{fig:tpert_412}, but for a temperature
    perturbation at 600 hPa, and with black circles showing the responses of
    \citeA{Kuang_JAS_2010} to a temperature perturbation at 565 hPa.}
    \label{fig:tpert_600}
\end{figure}

\begin{figure}[pth]
    \noindent\includegraphics[width=\textwidth]{figures/pert_diffs_T_0.5_@730}
    \caption{As in Figure \ref{fig:tpert_412}, but for a temperature
    perturbation at 730 hPa, and with black circles showing the responses of
    \citeA{Kuang_JAS_2010} to a temperature perturbation at 729 hPa.}
    \label{fig:tpert_730}
\end{figure}

\begin{figure}[pth]
    \noindent\includegraphics[width=\textwidth]{figures/pert_diffs_q_0.0002_@500}
    \caption{As in Figure \ref{fig:qpert_412}, but for a water vapor mixing
    ratio perturbation at 500 hPa, and with black circles showing the responses
    of \citeA{Kuang_JAS_2010} to a specific humidity perturbation at 483 hPa.}
    \label{fig:qpert_500}
\end{figure}

\begin{figure}[pth]
    \noindent\includegraphics[width=\textwidth]{figures/pert_diffs_q_0.0002_@600}
    \caption{As in Figure \ref{fig:qpert_412}, but for a perturbation at 600
    hPa, and with black circles showing the responses of \citeA{Kuang_JAS_2010}
    to a specific humidity perturbation at 565 hPa.}
    \label{fig:qpert_600}
\end{figure}

\begin{figure}[pth]
    \noindent\includegraphics[width=\textwidth]{figures/pert_diffs_q_0.0002_@730}
    \caption{As in Figure \ref{fig:qpert_412}, but for a perturbation at 730
    hPa, and with black circles showing the responses of \citeA{Kuang_JAS_2010}
    to a specific humidity perturbation at 729 hPa.}
    \label{fig:qpert_730}
\end{figure}

\begin{figure}[pth]
    \noindent\includegraphics[width=\textwidth]{figures/pert_var_T_0.5_@412}
    \caption{Mean responses (solid line) and the $\pm$ 1 standard deviation range (shaded) for a temperature perturbation of $\pm$0.5 K at 412 hPa in the WRF runs at 100 m grid spacing. The solid vertical line shows zero response, the dashed horizontal line shows the level of maximum perturbation.}
    \label{fig:var_T_412}
\end{figure}

\begin{figure}[pth]
    \noindent\includegraphics[width=\textwidth]{figures/pert_var_T_0.5_@850}
    \caption{As in Figure \ref{fig:var_T_412} but for a temperature perturbation at 850 hPa.}
    \label{fig:var_T_850}
\end{figure}

\begin{figure}[pth]
    \noindent\includegraphics[width=\textwidth]{figures/pert_var_q_0.0002_@412}
    \caption{As in Figure \ref{fig:var_T_412} but for a water vapor mixing ratio
    perturbation of 0.2 g kg$^{-1}$ at 412 hPa.}
    \label{fig:var_q_412}
\end{figure}

\begin{figure}[pth]
    \noindent\includegraphics[width=\textwidth]{figures/pert_var_q_0.0002_@850}
    \caption{As in Figure \ref{fig:var_q_412} but for a water vapor mixing ratio perturbation of 0.2 g kg$^{-1}$ at 850 hPa.}
    \label{fig:var_q_850}
\end{figure}

\end{document}
